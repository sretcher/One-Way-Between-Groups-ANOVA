\documentclass[]{article}
\usepackage{lmodern}
\usepackage{amssymb,amsmath}
\usepackage{ifxetex,ifluatex}
\usepackage{fixltx2e} % provides \textsubscript
\ifnum 0\ifxetex 1\fi\ifluatex 1\fi=0 % if pdftex
  \usepackage[T1]{fontenc}
  \usepackage[utf8]{inputenc}
\else % if luatex or xelatex
  \ifxetex
    \usepackage{mathspec}
  \else
    \usepackage{fontspec}
  \fi
  \defaultfontfeatures{Ligatures=TeX,Scale=MatchLowercase}
\fi
% use upquote if available, for straight quotes in verbatim environments
\IfFileExists{upquote.sty}{\usepackage{upquote}}{}
% use microtype if available
\IfFileExists{microtype.sty}{%
\usepackage{microtype}
\UseMicrotypeSet[protrusion]{basicmath} % disable protrusion for tt fonts
}{}
\usepackage[margin=1in]{geometry}
\usepackage{hyperref}
\hypersetup{unicode=true,
            pdftitle={One-Way Between-Groups ANOVA},
            pdfborder={0 0 0},
            breaklinks=true}
\urlstyle{same}  % don't use monospace font for urls
\usepackage{color}
\usepackage{fancyvrb}
\newcommand{\VerbBar}{|}
\newcommand{\VERB}{\Verb[commandchars=\\\{\}]}
\DefineVerbatimEnvironment{Highlighting}{Verbatim}{commandchars=\\\{\}}
% Add ',fontsize=\small' for more characters per line
\usepackage{framed}
\definecolor{shadecolor}{RGB}{248,248,248}
\newenvironment{Shaded}{\begin{snugshade}}{\end{snugshade}}
\newcommand{\AlertTok}[1]{\textcolor[rgb]{0.94,0.16,0.16}{#1}}
\newcommand{\AnnotationTok}[1]{\textcolor[rgb]{0.56,0.35,0.01}{\textbf{\textit{#1}}}}
\newcommand{\AttributeTok}[1]{\textcolor[rgb]{0.77,0.63,0.00}{#1}}
\newcommand{\BaseNTok}[1]{\textcolor[rgb]{0.00,0.00,0.81}{#1}}
\newcommand{\BuiltInTok}[1]{#1}
\newcommand{\CharTok}[1]{\textcolor[rgb]{0.31,0.60,0.02}{#1}}
\newcommand{\CommentTok}[1]{\textcolor[rgb]{0.56,0.35,0.01}{\textit{#1}}}
\newcommand{\CommentVarTok}[1]{\textcolor[rgb]{0.56,0.35,0.01}{\textbf{\textit{#1}}}}
\newcommand{\ConstantTok}[1]{\textcolor[rgb]{0.00,0.00,0.00}{#1}}
\newcommand{\ControlFlowTok}[1]{\textcolor[rgb]{0.13,0.29,0.53}{\textbf{#1}}}
\newcommand{\DataTypeTok}[1]{\textcolor[rgb]{0.13,0.29,0.53}{#1}}
\newcommand{\DecValTok}[1]{\textcolor[rgb]{0.00,0.00,0.81}{#1}}
\newcommand{\DocumentationTok}[1]{\textcolor[rgb]{0.56,0.35,0.01}{\textbf{\textit{#1}}}}
\newcommand{\ErrorTok}[1]{\textcolor[rgb]{0.64,0.00,0.00}{\textbf{#1}}}
\newcommand{\ExtensionTok}[1]{#1}
\newcommand{\FloatTok}[1]{\textcolor[rgb]{0.00,0.00,0.81}{#1}}
\newcommand{\FunctionTok}[1]{\textcolor[rgb]{0.00,0.00,0.00}{#1}}
\newcommand{\ImportTok}[1]{#1}
\newcommand{\InformationTok}[1]{\textcolor[rgb]{0.56,0.35,0.01}{\textbf{\textit{#1}}}}
\newcommand{\KeywordTok}[1]{\textcolor[rgb]{0.13,0.29,0.53}{\textbf{#1}}}
\newcommand{\NormalTok}[1]{#1}
\newcommand{\OperatorTok}[1]{\textcolor[rgb]{0.81,0.36,0.00}{\textbf{#1}}}
\newcommand{\OtherTok}[1]{\textcolor[rgb]{0.56,0.35,0.01}{#1}}
\newcommand{\PreprocessorTok}[1]{\textcolor[rgb]{0.56,0.35,0.01}{\textit{#1}}}
\newcommand{\RegionMarkerTok}[1]{#1}
\newcommand{\SpecialCharTok}[1]{\textcolor[rgb]{0.00,0.00,0.00}{#1}}
\newcommand{\SpecialStringTok}[1]{\textcolor[rgb]{0.31,0.60,0.02}{#1}}
\newcommand{\StringTok}[1]{\textcolor[rgb]{0.31,0.60,0.02}{#1}}
\newcommand{\VariableTok}[1]{\textcolor[rgb]{0.00,0.00,0.00}{#1}}
\newcommand{\VerbatimStringTok}[1]{\textcolor[rgb]{0.31,0.60,0.02}{#1}}
\newcommand{\WarningTok}[1]{\textcolor[rgb]{0.56,0.35,0.01}{\textbf{\textit{#1}}}}
\usepackage{longtable,booktabs}
\usepackage{graphicx,grffile}
\makeatletter
\def\maxwidth{\ifdim\Gin@nat@width>\linewidth\linewidth\else\Gin@nat@width\fi}
\def\maxheight{\ifdim\Gin@nat@height>\textheight\textheight\else\Gin@nat@height\fi}
\makeatother
% Scale images if necessary, so that they will not overflow the page
% margins by default, and it is still possible to overwrite the defaults
% using explicit options in \includegraphics[width, height, ...]{}
\setkeys{Gin}{width=\maxwidth,height=\maxheight,keepaspectratio}
\IfFileExists{parskip.sty}{%
\usepackage{parskip}
}{% else
\setlength{\parindent}{0pt}
\setlength{\parskip}{6pt plus 2pt minus 1pt}
}
\setlength{\emergencystretch}{3em}  % prevent overfull lines
\providecommand{\tightlist}{%
  \setlength{\itemsep}{0pt}\setlength{\parskip}{0pt}}
\setcounter{secnumdepth}{0}
% Redefines (sub)paragraphs to behave more like sections
\ifx\paragraph\undefined\else
\let\oldparagraph\paragraph
\renewcommand{\paragraph}[1]{\oldparagraph{#1}\mbox{}}
\fi
\ifx\subparagraph\undefined\else
\let\oldsubparagraph\subparagraph
\renewcommand{\subparagraph}[1]{\oldsubparagraph{#1}\mbox{}}
\fi

%%% Use protect on footnotes to avoid problems with footnotes in titles
\let\rmarkdownfootnote\footnote%
\def\footnote{\protect\rmarkdownfootnote}

%%% Change title format to be more compact
\usepackage{titling}

% Create subtitle command for use in maketitle
\newcommand{\subtitle}[1]{
  \posttitle{
    \begin{center}\large#1\end{center}
    }
}

\setlength{\droptitle}{-2em}

  \title{\textbf{One-Way Between-Groups ANOVA}}
    \pretitle{\vspace{\droptitle}\centering\huge}
  \posttitle{\par}
    \author{}
    \preauthor{}\postauthor{}
    \date{}
    \predate{}\postdate{}
  

\begin{document}
\maketitle

A clinical study was conducted to assess the effect of three
formulations of the same drug on reducing cholesterol. The formulations
were 20mg at once (1time), 10mg twice a day (2times), and 5mg four times
a day (4times). In addition, two competing drugs were used as a control
group (drugD and drugE). Fifty patients were randomly selected for the
study and 10 patients were randomly assigned to each treatment. The
purpose of the study was to find which of the formulations, if any, were
the most effective at reducing cholesterol and how these formulations
compare with the existing drugs. We will do this by conducting a One-Way
ANOVA F-Test to compare treatment means in a Completely Randomized
Design.

\hypertarget{examining-the-data}{%
\subsubsection{\texorpdfstring{\textbf{Examining the
Data}}{Examining the Data}}\label{examining-the-data}}

The following code loads the packages needed for the analysis.

\begin{Shaded}
\begin{Highlighting}[]
\KeywordTok{library}\NormalTok{(multcomp)}
\KeywordTok{library}\NormalTok{(tidyverse)}
\KeywordTok{library}\NormalTok{(rstatix)}
\KeywordTok{library}\NormalTok{(ggpubr)}
\KeywordTok{library}\NormalTok{(gplots)}
\KeywordTok{library}\NormalTok{(car)}
\end{Highlighting}
\end{Shaded}

The table below shows how some lesser known packages are used in the
analysis.

\begin{longtable}[]{@{}ll@{}}
\toprule
\begin{minipage}[b]{0.47\columnwidth}\raggedright
Package\strut
\end{minipage} & \begin{minipage}[b]{0.47\columnwidth}\raggedright
Usage in Project\strut
\end{minipage}\tabularnewline
\midrule
\endhead
\begin{minipage}[t]{0.47\columnwidth}\raggedright
\texttt{multcomp}\strut
\end{minipage} & \begin{minipage}[t]{0.47\columnwidth}\raggedright
Our data comes from the dataset \emph{cholesterol} in \texttt{multcomp}.
This package also provides methods for multiple mean comparisons between
treatments which will be used later in the analysis.\strut
\end{minipage}\tabularnewline
\begin{minipage}[t]{0.47\columnwidth}\raggedright
\texttt{rstatix}\strut
\end{minipage} & \begin{minipage}[t]{0.47\columnwidth}\raggedright
\texttt{rstatix} provides a framework that allows us to use statistics
with the tidyverse by providing pipeable statistical functions.\strut
\end{minipage}\tabularnewline
\begin{minipage}[t]{0.47\columnwidth}\raggedright
\texttt{ggpubr}/\texttt{gplots}\strut
\end{minipage} & \begin{minipage}[t]{0.47\columnwidth}\raggedright
\texttt{ggpubr} and \texttt{gplots} allow us to visualize data easier
instead of spending forever customizing \texttt{ggplot2} plots.\strut
\end{minipage}\tabularnewline
\begin{minipage}[t]{0.47\columnwidth}\raggedright
\texttt{car}\strut
\end{minipage} & \begin{minipage}[t]{0.47\columnwidth}\raggedright
\texttt{car} has functions which are useful for detecting outliers in
our data.\strut
\end{minipage}\tabularnewline
\bottomrule
\end{longtable}

Our data comes from the dataset \emph{cholesterol} in \texttt{multcomp}.
The following code coverts the dataset to a \texttt{tibble}, which makes
it easier to use the \texttt{tidyverse} with our data.

\begin{Shaded}
\begin{Highlighting}[]
\NormalTok{cholesterol <-}\StringTok{ }\KeywordTok{tibble}\NormalTok{(cholesterol)}
\end{Highlighting}
\end{Shaded}

The \texttt{head} function is used to get an idea of how the data is
stored. The levels of the factor \texttt{trt} represent the five
different drug formulations which are beleived to have an effect on our
response variable (cholesterol reduction).

\begin{Shaded}
\begin{Highlighting}[]
\KeywordTok{head}\NormalTok{(cholesterol)}
\end{Highlighting}
\end{Shaded}

\begin{verbatim}
## # A tibble: 6 x 2
##   trt   response
##   <fct>    <dbl>
## 1 1time     3.86
## 2 1time    10.4 
## 3 1time     5.91
## 4 1time     3.06
## 5 1time     7.72
## 6 1time     2.71
\end{verbatim}

To examine the treatments of the experiment, we use the function
\texttt{levels} on our factor \texttt{trt}. The treatments appear to be
in order and spelled correctly as stated in the introduction.

\begin{Shaded}
\begin{Highlighting}[]
\KeywordTok{levels}\NormalTok{(cholesterol}\OperatorTok{$}\NormalTok{trt)}
\end{Highlighting}
\end{Shaded}

\begin{verbatim}
## [1] "1time"  "2times" "4times" "drugD"  "drugE"
\end{verbatim}

The function \texttt{sample\_n\_by} is used to return two random rows
from each treatment. This gives us a better overview of data and all of
the treatments found by the \texttt{levels} function are included.

\begin{Shaded}
\begin{Highlighting}[]
\KeywordTok{as.data.frame}\NormalTok{(cholesterol) }\OperatorTok\StringTok{ }
\KeywordTok{sample_n_by}\NormalTok{(trt, }\DataTypeTok{size =} \DecValTok{2}\NormalTok{)}
\end{Highlighting}
\end{Shaded}

\begin{verbatim}
## # A tibble: 10 x 2
##    trt    response
##    <fct>     <dbl>
##  1 1time      3.06
##  2 1time      2.71
##  3 2times     8.60
##  4 2times    15.8 
##  5 4times    14.0 
##  6 4times     9.03
##  7 drugD     20.0 
##  8 drugD     18.0 
##  9 drugE     21.6 
## 10 drugE     17.3
\end{verbatim}

We use the \texttt{table} function to find the sample size for each
treatment. In general, when sample sizes are unequal across treatments,
ANOVA becomes more complicated.

Having unequal sample sizes can affect the homogeneity of variance
assumption which will result in a lower statistical power for the test.
Looking below, we see that our experiment has a balanced design with 10
experimental units for each treatment.

\begin{Shaded}
\begin{Highlighting}[]
\KeywordTok{table}\NormalTok{(cholesterol}\OperatorTok{$}\NormalTok{trt)}
\end{Highlighting}
\end{Shaded}

\begin{verbatim}
## 
##  1time 2times 4times  drugD  drugE 
##     10     10     10     10     10
\end{verbatim}

The function \texttt{summary} is used to view the distribution of our
response variable.

\begin{Shaded}
\begin{Highlighting}[]
\KeywordTok{summary}\NormalTok{(cholesterol}\OperatorTok{$}\NormalTok{response)}
\end{Highlighting}
\end{Shaded}

\begin{verbatim}
##    Min. 1st Qu.  Median    Mean 3rd Qu.    Max. 
##   2.304   8.409  12.605  12.738  17.519  27.244
\end{verbatim}

The following code computes the mean, standard deviation, and sample
size for each treatment.

\begin{Shaded}
\begin{Highlighting}[]
\NormalTok{group_stats <-}\StringTok{ }\NormalTok{cholesterol }\OperatorTok
\StringTok{              }\KeywordTok{group_by}\NormalTok{(trt) }\OperatorTok
\StringTok{              }\KeywordTok{summarise}\NormalTok{(}\DataTypeTok{n=}\KeywordTok{n}\NormalTok{(),}\DataTypeTok{group_mean =} \KeywordTok{mean}\NormalTok{(response),}\DataTypeTok{group_sd =} \KeywordTok{sd}\NormalTok{(response))}

\NormalTok{group_stats}
\end{Highlighting}
\end{Shaded}

\begin{verbatim}
## # A tibble: 5 x 4
##   trt        n group_mean group_sd
##   <fct>  <int>      <dbl>    <dbl>
## 1 1time     10       5.78     2.88
## 2 2times    10       9.22     3.48
## 3 4times    10      12.4      2.92
## 4 drugD     10      15.4      3.45
## 5 drugE     10      20.9      3.35
\end{verbatim}

\hypertarget{conditions-required-for-a-valid-anova-f-test-completely-randomized-design}{%
\subsubsection{\texorpdfstring{\textbf{Conditions Required for a Valid
Anova F-Test: Completely Randomized
Design}}{Conditions Required for a Valid Anova F-Test: Completely Randomized Design}}\label{conditions-required-for-a-valid-anova-f-test-completely-randomized-design}}

\begin{enumerate}
\def\labelenumi{\arabic{enumi}.}
\tightlist
\item
  The samples are randomly selected in an independent manner from the
  treatment populations.
\item
  All k sampled populations have distribution that are approximately
  normal.
\item
  The k population variances are equal.
\end{enumerate}

A patient is the experimental unit for this experiment. For a Completely
Randomized Design, independent random samples of experimental units are
selected for each treatment or k treatments can be randomly assigned to
a random sample of experimental units. Our experiment has the latter,
since 50 patients were randomly selected and than randomly assigned to
treatments. We will consider the first assumption to be fulfilled.

For the second assumption, we can either check the normality of the
residuals of our ANOVA model or we can check the normality of each
treatment population. The following code uses the function
\texttt{ggqqplot} function to plot a Q-Q plot of the response variable
for each treatment. Looking at the plots, we can see that the k
populations are normally distributed.

\begin{Shaded}
\begin{Highlighting}[]
\KeywordTok{ggqqplot}\NormalTok{(cholesterol, }\StringTok{"response"}\NormalTok{, }\DataTypeTok{facet.by =} \StringTok{"trt"}\NormalTok{)}
\end{Highlighting}
\end{Shaded}

\includegraphics{anova_files/figure-latex/unnamed-chunk-9-1.pdf}

Recall that Anova and Regression are both special cases of the general
linear model, which is why we can use the same \texttt{lm} function for
regression to find the residuals for ANOVA. We can then conduct the
Shapiro--Wilk test on the model's residuals to test for normality. Since
the p-value is larger than our significance level(0.05), we do not
reject the null hypothesis which means our k treatment populations are
most likely normal.

\begin{Shaded}
\begin{Highlighting}[]
\NormalTok{model <-}\StringTok{ }\KeywordTok{lm}\NormalTok{(response }\OperatorTok{~}\StringTok{ }\NormalTok{trt,}\DataTypeTok{data =}\NormalTok{ cholesterol)}
\KeywordTok{shapiro.test}\NormalTok{(model}\OperatorTok{$}\NormalTok{residuals)}
\end{Highlighting}
\end{Shaded}

\begin{verbatim}
## 
##  Shapiro-Wilk normality test
## 
## data:  model$residuals
## W = 0.98864, p-value = 0.9094
\end{verbatim}

The following code make a Q-Q plot of the above model's residuals.

\begin{Shaded}
\begin{Highlighting}[]
\KeywordTok{qqnorm}\NormalTok{(model}\OperatorTok{$}\NormalTok{residuals)}
\KeywordTok{qqline}\NormalTok{(model}\OperatorTok{$}\NormalTok{residuals)}
\end{Highlighting}
\end{Shaded}

\includegraphics{anova_files/figure-latex/unnamed-chunk-11-1.pdf}

The following code computes the Shapiro-Wilk test for each treatment.
Since we have large p-values, we have enough evidence to prove that the
treatment populations are normal.

\begin{Shaded}
\begin{Highlighting}[]
\NormalTok{cholesterol }\OperatorTok
\StringTok{  }\KeywordTok{group_by}\NormalTok{(trt) }\OperatorTok
\StringTok{  }\KeywordTok{summarize}\NormalTok{(}\DataTypeTok{W=}\KeywordTok{shapiro.test}\NormalTok{(response)}\OperatorTok{$}\NormalTok{statistic, }\DataTypeTok{p_value=}\KeywordTok{shapiro.test}\NormalTok{(response)}\OperatorTok{$}\NormalTok{p.value)}
\end{Highlighting}
\end{Shaded}

\begin{verbatim}
## # A tibble: 5 x 3
##   trt        W p_value
##   <fct>  <dbl>   <dbl>
## 1 1time  0.931   0.454
## 2 2times 0.943   0.589
## 3 4times 0.955   0.726
## 4 drugD  0.934   0.489
## 5 drugE  0.981   0.970
\end{verbatim}

The following code does the same thing as the code above but it is a lot
cleaner. The \texttt{shapiro\_test} function is from the package
\texttt{rstatix} which allows us to use the pipe with statistical tests.

\begin{Shaded}
\begin{Highlighting}[]
\NormalTok{cholesterol }\OperatorTok
\StringTok{  }\KeywordTok{group_by}\NormalTok{(trt) }\OperatorTok
\StringTok{  }\KeywordTok{shapiro_test}\NormalTok{(response)}
\end{Highlighting}
\end{Shaded}

\begin{verbatim}
## # A tibble: 5 x 4
##   trt    variable statistic     p
##   <fct>  <chr>        <dbl> <dbl>
## 1 1time  response     0.931 0.454
## 2 2times response     0.943 0.589
## 3 4times response     0.955 0.726
## 4 drugD  response     0.934 0.489
## 5 drugE  response     0.981 0.970
\end{verbatim}

Looking at all the code above, we will consider the second assumption to
be fulfilled. For the third assumptions we can use Bartlett's Test of
Homogeneity of Variances to find out if the k population variances are
equal. We are using Bartlett's test here instead of Levene's test
because Bartlett's test performs better when we know that our data is
normal. Since the p-value is large for Bartlett's test, the variances
among treatments do not differ significantly.

\begin{Shaded}
\begin{Highlighting}[]
\KeywordTok{bartlett.test}\NormalTok{(response}\OperatorTok{~}\NormalTok{trt,}\DataTypeTok{data =}\NormalTok{ cholesterol)}
\end{Highlighting}
\end{Shaded}

\begin{verbatim}
## 
##  Bartlett test of homogeneity of variances
## 
## data:  response by trt
## Bartlett's K-squared = 0.57975, df = 4, p-value = 0.9653
\end{verbatim}

Outliers inflate the MST and MSE, which leads to a smaller F Statistic.
This means that the chance of rejecting the null hypothesis is also
lowered. It's important to make sure there are no data collection errors
and to see the effect of outliers on our analysis. Using the
\texttt{identify\_outliers} function allows us to identify outliers
using boxplot methods. Looking below, we do not have any extreme
outliers so we will keep the data the same.

\begin{Shaded}
\begin{Highlighting}[]
\NormalTok{cholesterol }\OperatorTok\StringTok{ }
\StringTok{  }\KeywordTok{group_by}\NormalTok{(trt) }\OperatorTok
\StringTok{  }\KeywordTok{identify_outliers}\NormalTok{(response)}
\end{Highlighting}
\end{Shaded}

\begin{verbatim}
## # A tibble: 3 x 4
##   trt    response is.outlier is.extreme
##   <fct>     <dbl> <lgl>      <lgl>     
## 1 2times    13.6  TRUE       FALSE     
## 2 2times     3.51 TRUE       FALSE     
## 3 2times    15.8  TRUE       FALSE
\end{verbatim}

\hypertarget{hypotheses-for-anova-f-test-to-compare-k-treatment-means}{%
\subsubsection{\texorpdfstring{\textbf{Hypotheses for ANOVA F-Test to
Compare k Treatment
Means}}{Hypotheses for ANOVA F-Test to Compare k Treatment Means}}\label{hypotheses-for-anova-f-test-to-compare-k-treatment-means}}

We will denote the population means of k treatments as u1, u2,
u3,\ldots{} and so on. We will test the null hypothesis that treatment
means are equal against the alternative hypothesis that at least two
treatment means differ. We will use a level of significance of .05 for
this test.

Ho: u1 = u2 = u3 = u4 = u5

Ha: At least two of these treatment means differ.

\hypertarget{conducting-the-test}{%
\subsubsection{\texorpdfstring{\textbf{Conducting the
Test}}{Conducting the Test}}\label{conducting-the-test}}

To conduct a one-way ANOVA test, we use the \texttt{aov} function which
requires a formula and a dataset. The formula below tells R to conduct
the ANOVA test where response is the response variable and trt is the
factor. We use the \texttt{summary} function on \texttt{fit} to
summarize the results from the ANOVA tets.

Looking at the results, we see that our F test statistic is quite big
and our p-value is small. Recall that the F statistic is the average
variation between treatment means over the average variation within
treatments. The bigger the F statistic is, the more likely there is
statistically significant difference between means. Since the p-value is
smaller than our alpha and the F statistic is quite large, we can reject
the null hypotheses and we have enough evidence that to prove that the
treatment means are not equal.

\begin{Shaded}
\begin{Highlighting}[]
\NormalTok{fit <-}\StringTok{ }\KeywordTok{aov}\NormalTok{(response }\OperatorTok{~}\StringTok{ }\NormalTok{trt,}\DataTypeTok{data=}\NormalTok{cholesterol)}
\KeywordTok{summary}\NormalTok{(fit)}
\end{Highlighting}
\end{Shaded}

\begin{verbatim}
##             Df Sum Sq Mean Sq F value   Pr(>F)    
## trt          4 1351.4   337.8   32.43 9.82e-13 ***
## Residuals   45  468.8    10.4                     
## ---
## Signif. codes:  0 '***' 0.001 '**' 0.01 '*' 0.05 '.' 0.1 ' ' 1
\end{verbatim}

Since we rejected the null hypothesis, we know that at least two
treatment means differ but we are not sure which treatments differ
significantly between each other. To make multiple comparisons of
treatment means, we will use Tukey's method which works best with equal
sample sizes and pairwise comparisons.

Before we use Tukey's Method, recall that two independent sample t-tests
are equivalent to two independent sample f-tests. Because of this, we
can construct a one sample 95\% confidence interval for each treatment.
If two treatment's confidence intervals intersect, they are not
significantly different.

The \texttt{plotmeans} function plots the mean and ci for each
treatment. From this plot, we see that 4times is the best drug
formulation and is significantly different from 1time. We also see that
drugE is statistically better different than the other treatments. We
also see that 2times is not signifcantly different from 1times or
4times. Finally we see that DrugD is statistically different from 2times
, 1time, and drugE but not statistically different from 4times.

\begin{Shaded}
\begin{Highlighting}[]
\KeywordTok{plotmeans}\NormalTok{(cholesterol}\OperatorTok{$}\NormalTok{response}\OperatorTok{~}\NormalTok{cholesterol}\OperatorTok{$}\NormalTok{trt, }\DataTypeTok{ci.label =} \OtherTok{TRUE}\NormalTok{,}\DataTypeTok{n.label =} \OtherTok{FALSE}\NormalTok{)}
\end{Highlighting}
\end{Shaded}

\includegraphics{anova_files/figure-latex/unnamed-chunk-17-1.pdf}

The following code shows the confidence intervals above are two-sided
t-test confidence intervals. The two-sided t-test confidence interval
for the treatment 1time is the same as the one above.

\begin{Shaded}
\begin{Highlighting}[]
\NormalTok{g <-}\StringTok{ }\KeywordTok{filter}\NormalTok{(cholesterol,trt}\OperatorTok{==}\StringTok{"1time"}\NormalTok{)}
\KeywordTok{t.test}\NormalTok{(}\DataTypeTok{x =}\NormalTok{g}\OperatorTok{$}\NormalTok{response, }\DataTypeTok{conf.level =} \FloatTok{.95}\NormalTok{, }\DataTypeTok{alternative =} \StringTok{"two.sided"}\NormalTok{)}
\end{Highlighting}
\end{Shaded}

\begin{verbatim}
## 
##  One Sample t-test
## 
## data:  g$response
## t = 6.3528, df = 9, p-value = 0.0001324
## alternative hypothesis: true mean is not equal to 0
## 95 percent confidence interval:
##  3.723092 7.840848
## sample estimates:
## mean of x 
##   5.78197
\end{verbatim}

Why don't we just conduct multiple t-tests to find the confidence
intervals between groups? The problem with this and the approach above
is that it increases the likelihood of a Type 1 error. Using Tukey's
method, we can say that collectively 95\% of the intervals include the
true difference between groups. This is because Tukey's method adjusts
for the amount of groups being compared unlike the t-test.

Looking below, any p-value below our significance level means those two
groups are significant different. A 95\% confidence interval is also
provided which contains the true difference between those groups. For
example, we that drugE-drugD has a low p-value meaning they are
significantly different. Looking at the ci, we see that the true mean of
drugE is (1.48,9.68) higher than drugd.

\begin{Shaded}
\begin{Highlighting}[]
\KeywordTok{TukeyHSD}\NormalTok{(fit)}
\end{Highlighting}
\end{Shaded}

\begin{verbatim}
##   Tukey multiple comparisons of means
##     95% family-wise confidence level
## 
## Fit: aov(formula = response ~ trt, data = cholesterol)
## 
## $trt
##                   diff        lwr       upr     p adj
## 2times-1time   3.44300 -0.6582817  7.544282 0.1380949
## 4times-1time   6.59281  2.4915283 10.694092 0.0003542
## drugD-1time    9.57920  5.4779183 13.680482 0.0000003
## drugE-1time   15.16555 11.0642683 19.266832 0.0000000
## 4times-2times  3.14981 -0.9514717  7.251092 0.2050382
## drugD-2times   6.13620  2.0349183 10.237482 0.0009611
## drugE-2times  11.72255  7.6212683 15.823832 0.0000000
## drugD-4times   2.98639 -1.1148917  7.087672 0.2512446
## drugE-4times   8.57274  4.4714583 12.674022 0.0000037
## drugE-drugD    5.58635  1.4850683  9.687632 0.0030633
\end{verbatim}

The following code shows another way to compute tukey's method for
comparing treatments.

\begin{Shaded}
\begin{Highlighting}[]
\NormalTok{pwc <-}\StringTok{ }\NormalTok{cholesterol }\OperatorTok\StringTok{ }\KeywordTok{tukey_hsd}\NormalTok{(response}\OperatorTok{~}\NormalTok{trt)}
\NormalTok{pwc}
\end{Highlighting}
\end{Shaded}

\begin{verbatim}
## # A tibble: 10 x 8
##    term  group1 group2 estimate conf.low conf.high    p.adj p.adj.signif
##  * <chr> <chr>  <chr>     <dbl>    <dbl>     <dbl>    <dbl> <chr>       
##  1 trt   1time  2times     3.44   -0.658      7.54 1.38e- 1 ns          
##  2 trt   1time  4times     6.59    2.49      10.7  3.54e- 4 ***         
##  3 trt   1time  drugD      9.58    5.48      13.7  3.46e- 7 ****        
##  4 trt   1time  drugE     15.2    11.1       19.3  1.35e-12 ****        
##  5 trt   2times 4times     3.15   -0.951      7.25 2.05e- 1 ns          
##  6 trt   2times drugD      6.14    2.03      10.2  9.61e- 4 ***         
##  7 trt   2times drugE     11.7     7.62      15.8  2.27e- 9 ****        
##  8 trt   4times drugD      2.99   -1.11       7.09 2.51e- 1 ns          
##  9 trt   4times drugE      8.57    4.47      12.7  3.72e- 6 ****        
## 10 trt   drugD  drugE      5.59    1.49       9.69 3.06e- 3 **
\end{verbatim}

The following code visualizes the confidence intervals of differences
between group means. If a confidence interval includes 0, those groups
are not statistically different from each other.

\begin{Shaded}
\begin{Highlighting}[]
\KeywordTok{par}\NormalTok{(}\DataTypeTok{las=}\DecValTok{2}\NormalTok{,}\DataTypeTok{mar =} \KeywordTok{c}\NormalTok{(}\DecValTok{7}\NormalTok{,}\DecValTok{7}\NormalTok{,}\DecValTok{7}\NormalTok{,}\DecValTok{7}\NormalTok{))}
\KeywordTok{plot}\NormalTok{(}\KeywordTok{TukeyHSD}\NormalTok{(fit))}
\end{Highlighting}
\end{Shaded}

\includegraphics{anova_files/figure-latex/unnamed-chunk-21-1.pdf}

Finally, we use the following code which gives us the boxplots for each
treatment along with Tukey's compact letter ranking. If two boxplots
have the same letter, they are not statistically different. Overall,
4times is the best formulation for a certain drug for reducing
cholesterol but falls short when compared to the drug E.

\begin{Shaded}
\begin{Highlighting}[]
\KeywordTok{par}\NormalTok{(}\DataTypeTok{las=}\DecValTok{1}\NormalTok{,}\DataTypeTok{mar=}\KeywordTok{c}\NormalTok{(}\DecValTok{6}\NormalTok{,}\DecValTok{6}\NormalTok{,}\DecValTok{6}\NormalTok{,}\DecValTok{6}\NormalTok{))}
\NormalTok{tuk <-}\StringTok{ }\KeywordTok{glht}\NormalTok{(fit,}\DataTypeTok{linfct=}\KeywordTok{mcp}\NormalTok{(}\DataTypeTok{trt=}\StringTok{"Tukey"}\NormalTok{))}
\KeywordTok{plot}\NormalTok{(}\KeywordTok{cld}\NormalTok{(tuk,}\DataTypeTok{level =} \FloatTok{.05}\NormalTok{),}\DataTypeTok{col =} \StringTok{"lightblue"}\NormalTok{)}
\end{Highlighting}
\end{Shaded}

\includegraphics[height=20]{anova_files/figure-latex/unnamed-chunk-22-1}


\end{document}
